\documentclass[10pt]{memoir}

% based on kieran healy's memoir modifications
\usepackage{mako-mem}
\chapterstyle{article-3}
\pagestyle{memo}

\usepackage{ucs}
\usepackage[utf8x]{inputenc}

\usepackage[T1]{fontenc}
\usepackage{textcomp}
\usepackage[garamond]{mathdesign}

\usepackage[letterpaper,left=1.2in,right=1.2in,top=1.2in,bottom=1.2in]{geometry}

% packages i use in essentially every document
\usepackage{graphicx}
\usepackage{enumerate}

% packages i use in many documents but leave off by default
% \usepackage{amsmath, amsthm, amssymb}
% \usepackage{dcolumn}
% \usepackage{endfloat}

% import and customize urls
\usepackage[usenames,dvipsnames]{color}
\usepackage[breaklinks]{hyperref}

\hypersetup{colorlinks=true, linkcolor=Black, citecolor=Black, filecolor=Blue,
    urlcolor=Blue, unicode=true}

% add bibliographic stuff 
\usepackage[round]{natbib}
\def\citepos#1{\citeauthor{#1}'s (\citeyear{#1})}
\def\citespos#1{\citeauthor{#1}' (\citeyear{#1})}

% import vc stuff after running `make vc`: \input{vc} \pagestyle{kjhgit}

\newenvironment{enumerate*}%
  {\begin{enumerate}%
    \setlength{\itemsep}{0pt}%
    \setlength{\parskip}{0pt}}%
  {\end{enumerate}}

\begin{document}

\setlength{\parskip}{4.5pt}

\baselineskip 14.5pt

\title{Teaching Statement}
\author{Benjamin Mako Hill}
\date{}

% \published{\textsc{\textcolor{BrickRed}{This document is an
% unpublished draft.\\ Please do not distribute or cite without
% permission.}}}

\maketitle

Graduating PhD students have spent most of their lives in
apprenticeship relationships. From their first day of grade school to
their dissertation defense, students learn everything from reading and
arithmetic to sociological theory and multi-level statistical modeling
from teachers who use that knowledge themselves.  ``I know something
that I find useful,'' a teacher might say, ``and I want my student to
be like me.''

In much of higher education -- and in graduate and professional
teaching in particular -- this relationship breaks down. In business
schools, where I teach most often, lectures are given by professors
trained as academic sociologists, economists, and psychologists. Very
few MBAs become social scientists. I have seen how a failure to
recognize this dynamic can lead to a lack of respect and connection
between teachers and students treated as, ``the folks who pay the
bills.''

Business school has also shown me that teaching that overcomes this
dynamic can lead to transformative learning. Teaching across
intellectual domains goes beyond the reproduction of skills and
knowledge and becomes the creation of new knowledge in the context of
students' personal experiences. I understand that most of my students
will not become a researcher like me. I believe that in spite of this
challenging relationship, and because of it, I can teach students in
ways that surprise, connect, and enrich. In my teaching, I address
this dynamic in three different ways.

First, I strive to make my teaching material relevant to my students
experiences and interests. I always seek to communicate why the
material I teach is relevant and how it will be useful. I have taught
identical material to engineers, MBAs, and executives and have worked
to refine and tailor my message for each audience.

Second, I attempt to involve students directly in learning. Even in
large lectures, I engage students interactively in discussion of
examples from their experience and adapt my teaching to emphasize
relevant material. In assignments, I challenge students to integrate
course concepts with their experience and interests.

Third, and most importantly, I structure my teaching around an
explicit mutual respect. Before each lecture, I reflect on the total
student-hours my teaching will consume. I realize that in every class
meeting, my students give me dozens, even hundreds, of hours of their
attention. I strive to never waste it. I continually seek feedback
from my students so that my teaching is more relevant, useful, and
important to them.

\section{Teaching Experience}

Over the course of graduate school, I have learned to teach from my
mentors and have put this philosophy into practice in lectures and
seminars to MBAs, engineers, executives, undergraduates, and Masters
of Science students.

Over the last three years, I have served as the teaching assistant for
Professor Eric von Hippel's lecture courses on innovation where I have
worked closely with students on the design and evaluation of their
course projects. In these classes, I have developed, delivered, and
refined a series of ninety-minute lectures as a guest lecturer. These
include a lecture on online innovation communities using the case of
consumer ``hacking'' of Canon cameras and a practical lecture on how
to attract participants to online communities.

After positive evaluations from students, I have been invited to give
regular lectures in MIT's Executive Education and Visiting MBA
programs. These lectures have focused on introducting concepts on
management of innovation and user communities and on practical methods
for putting these into action including lead user methods, broadcast
search, and the construction of user communities.

In addition to experience lecturing, I have also run a series of
seminars for smaller groups of graduate students. Working with Tom
Malone at the Center for Collective Intelligence, I coordinated an
interdisciplinary seminar on collective intelligence. Working with
Chris Csikszentmihályi, I organized and ran a graduate seminar on
Free, Libre and Open Source Software.

Outside of organizing my own seminars, I have guest-taught in a number
of seminars at MIT Sloan, the MIT Media Lab, the MIT Program on
Comparative Media Studies, Harvard Law School, the Stanford Design
School, and elsewhere. Since 2011, I have also coordinated a reading
group on empirical research into online cooperation at the Berkman
Center for Internet and Society at Harvard.

\section{Mentoring}

Of course, not all of teaching is unlike apprenticeship and I have
also enjoyed my experience as a mentor to developing scholars and
researchers. I have had the pleasure of mentoring several
undergraduates at MIT through the Undergraduate Research Opportunities
Program. These students worked with me on both a full-time basis over
the summer and in a part-time capacity over the academic year giving
me experience both in day-to-day management and more hands-off
relationships.

Additionally, I have served as an external advisor to two Masters
degree students. I advised and evaluated one thesis on technology
design and in am currently advising a social scientific analysis of a
large free software community. In both cases, I have enjoyed meeting
regularly and engaging with students over the course of their research
projects.

\section{Example Courses}

Undergraduate ---

\begin{enumerate*}
\item \emph{Innovation in the Internet Age}: An introduction to the
  theory and practice of innovation management. Topics include
  traditional firm-based innovation as well innovation by hackers,
  user communities, free and open source software, and lead users.
\item \emph{Quantitative Research Methods}: An introductory class on
  applied statistics. Topics include basic statistical methods up to,
  and including, linear regression with programming exercises using
  real data.
\item \emph{Computer Mediated Communication}: An overview of practical
  and theoretical issues related to computer-mediated
  communication. The class focuses on analyses of practice but also
  incorporates readings and lectures on system implementation and
  design.
\end{enumerate*}

Graduate ---
\begin{enumerate*}
\item \emph{Topics in Peer Production}: Seminar on foundational work
  as well as recent advances in the study and support of free and open
  source software, wikis, and remixing communities.
\item \emph{Research Methods for ``Big Data''}: An introduction to
  statistical methods and tools for finding and manipulating very
  large datasets. Topics include network analysis, analysis of
  unstructured text, and programming for massively parallel computing
  systems.
\item \emph{Social Computing}: The theory, analysis, and design of
  large scale, computer-mediated social systems. Final projects will
  challenge students to create a new systems or execute a study of an
  existing community.
\end{enumerate*}



\end{document}

