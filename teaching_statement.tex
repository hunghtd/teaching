\documentclass[10pt]{memoir}

% based on kieran healy's memoir modifications
\usepackage{mako-mem}
\chapterstyle{article-3}
\pagestyle{memo}

\usepackage{ucs}
\usepackage[utf8x]{inputenc}

\usepackage[T1]{fontenc}
\usepackage{textcomp}
\usepackage[garamond]{mathdesign}

\usepackage[letterpaper,left=1.2in,right=1.2in,top=1.2in,bottom=1.2in]{geometry}

% packages i use in essentially every document
\usepackage{graphicx}
\usepackage{enumerate}

% packages i use in many documents but leave off by default
% \usepackage{amsmath, amsthm, amssymb}
% \usepackage{dcolumn}
% \usepackage{endfloat}

% import and customize urls
\usepackage[usenames,dvipsnames]{color}
\usepackage[breaklinks]{hyperref}

\hypersetup{colorlinks=true, linkcolor=Black, citecolor=Black, filecolor=Blue,
    urlcolor=Blue, unicode=true}

% add bibliographic stuff 
\usepackage[round]{natbib}
\def\citepos#1{\citeauthor{#1}'s (\citeyear{#1})}
\def\citespos#1{\citeauthor{#1}' (\citeyear{#1})}

% import vc stuff after running `make vc`: \input{vc} \pagestyle{kjhgit}

\newenvironment{enumerate*}%
  {\begin{enumerate}%
    \setlength{\itemsep}{0pt}%
    \setlength{\parskip}{0pt}}%
  {\end{enumerate}}

\begin{document}

\setlength{\parskip}{4.5pt}

\baselineskip 14.5pt

\title{Teaching Statement}
\author{Benjamin Mako Hill}
\date{}

% \published{\textsc{\textcolor{BrickRed}{This document is an
% unpublished draft.\\ Please do not distribute or cite without
% permission.}}}

\maketitle

When I was eighteen and frusterated with high school, I took extra
classes, graduated early, and moved to Ethiopia. A year later, I
matriculated at Hampshire College: an experimental institution without
grades, tests, or majors. I chose Hampshire becase I cared deeply
about a personal connection to learning that I felt more traditional
institutions would not afford.

Today, I am passioante about teaching and I take pride in teaching
well. However, as someone once driven away from traditional higher
education, I also have a healthy ambivalence about my role at the
front of the lecture hall and seminar table and strong feelings about
how to help students learn.  Before each lecture, I reflect on the
total human-hours my teaching consumes. In every class meeting, my
students give me dozens, even hundreds, of hours of their attention. I
strive to never waste it.

I have noted that graduating PhD students have spent most of their
lives in apprentice-like relationships. From their first day of grade
school to their dissertation defense, students learn eveything from
reading and arithmetic to sociological theory and multi-level
statistical modeling from teachers who have and use that knowledge
themselves.  ``I know something that I find useful,'' a teacher might
say, ``and I want my student to be like me.''

In much of higher education -- and in graduate and professional
teaching in particular -- this relationship breaks down for the first
time in most students' and teachers' lives. In business schools, where
I teach most often, lectures are given by professors trained as
academic sociologists, economists, and psychologists.  To say that few
business school students have an interest in becoming social
scientists would be understatement. I have seen how a failure to
recognize this dynamic can lead to a lack of respect and a lack of
connection between teachers and students seen as, ``the folks who pay
the bills.''

But this setting has also shown me that teaching that confronts, and
takes advantage of, this dynamic can lead to transformative learning
experiences. Successful teaching across intellectual domains goes
beyond the simple reproduction of skills and knowledge and becomes a
process of adapation and instantiation of knowledge in the context of
students' personal experiences. I understand that most of my students
do not want to be a researcher like me. I believe that in spite of
this unusual and challenging relationship, and \emph{because of it}, I
can teach students in ways that suprise, connect, and enrich.

\section{Teaching Experience}

Over the course of graduate school, I have learned to teach from my
mentors and have put this philosophy into practice in lectures and
seminars to MBAs, engineers, executives, undergraduates, and Masters
of Science students.

Over the last three years, I have served as the teaching assistant for
Professor Eric von Hippel's lecture courses on innovation where I have
worked closely with students on the design and evaluation of their
course projects. In these classes, I have developed, delivered, and
refined a series of 90 minutes lectures as a guest lecturer in those
classes. In particular, I have developed a lecture on Internet-based
user innovation communities based around the case of consumer
``hacking'' of Canon cameras and a practical lecture on how to attract
participants to online communities.

After positive evaluations from students in these course, I have been
invited to give regular lectures in MIT's Executive Education and
Visiting MBA programs. These lectures have focused on fundemental
introduction to concepts on innovation management and user communities
and on practical methods for putting these into action including lead
user methods, broadcast search, and the construction of user
communities.

In addition to experience in the lecture hall, I have also run a
series of seminars for smaller groups of graduate students. Working
with Tom Malone at the Center for Collective Intelligence, I
coordinated an interdisciplinary seminar on collective
intelligence. Working with Chris Csikszentmihályi, I organized and ran
a graduate seminar on Free, Libre and Open Source
Software.

Outside of organizing my own seminars, I have taught in a number of
seminars at MIT Sloan, the MIT Media Lab, the MIT Program on
Comparative Media Studies, Harvard Law School, the Stanford Design
School, and elsewhere. Since 2011, I have also coordinated a reading
group on empirical research into online cooperation at the Berkman
Center for Intenet and Society at Harvard.

\section{Mentoring}

Of course, not all of teaching is unlike apprenticeship and I have
also enjoyed my experience as a mentor to developing scholars and
researchers. I have had the pleasure of mentoring several
undergraduates at MIT through the Undergraduate Research Opportunities
Program. These students worked with me on both a full-time basis over
the summer and in a part-time capacity over the academic year giving
me experience with day-to-day management and more hands-off
relationships.

Additionally, I have served as an external advisor to two Masters
degree students. I advised and evaluated one Masters Thesis on
technology design and in am currently advising a Masters thesis
studying a large free software community. In both cases, I have
enjoyed meeting regularly and engaging with students over the life of
their research projects.

\section{Example Courses}

Undergraduate ---

\begin{enumerate*}
\item \emph{Innovation in the Internet Age}: An introduction to the
  theory and practice of innovation management. Topics include
  traditional firm-based innovation as well innovation by hackers,
  user communities, free and open source software, and lead users.
\item \emph{Quantitative Research Methods}: An introductory class on
  applied statistics. Topics include basic stastical methods up to,
  and including, linear regression with programming excercises using
  real data.
\item \emph{Computer Mediated Communication}: An overview of practical
  and theoretical issues related to computer-mediated
  communication. The class focuses on analyses of pratice but also
  incorporate readings and lectures on system implementation and
  design.
\end{enumerate*}

Graduate ---
\begin{enumerate*}
\item \emph{Topics in Peer Production}: Seminar on foundational work
  as well as recent advances in the study and support of free and open
  source software, wikis, and remixing communities.
\item \emph{Research Methods for ``Big Data''}: An introduction to
  statistical methods and tools for finding and manipulating very
  large datasets. Topics include network analysis, analysis of
  unstructured text, and programming for massively parallel computing
  systems.
\item \emph{Social Computing}: The theory, analysis, and design of
  large scale, computer mediated social systems. Final projects will
  challenge students to create a new systems or execute a study of an
  existing system.
\end{enumerate*}



\end{document}

